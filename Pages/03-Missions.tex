\chapter{Les Missions réalisées}
\section{Enjeux et cadre des missions effectuées}
Ma première mission était la réalisation d'une application mobile multi-plateforme 
reprennant les fonctionnalités du site web client tout en prenant en compte 
les différences d'expérience utilisateur qu'apporte une interface tactile.\newline

Les technologies utilisées pour cette mission sont les suivantes:
\begin{itemize}
    \item pour le site client: C\# + webforms asp.net, JavaScript + JQuery, serveur Web IIS 
    dans un premier temp, la seconde mission est le développement d'un nouveau site client.
	\item pour l'application mobile: TypeScript + cordova + Ionic 3 \newline
\end{itemize} 

La seconde mission était la refonte totale du site web client qui contient aussi l'API pour 
l'application mobile, l'objectif 

\section{Focus sur les missions les plus complexes}


\newpage
\section{Bilan et recul sur les missions}

Sur le plan professionnel cette année a été très enrichissante, j'ai pu renforcer mes compétences dans 
les framework Angular, Ionic et .NET que j'avais commencé à acquérir lors de mon année de licence 
professionnelle, sur le projet de site web j'ai pu appliquer pleinement mes compétences 
dans l'architecture des logiciels avec la mise en place de méthodes de développement et 
de gestion de projet agile dans un premier temp puis avec l'application des bonnes pratiques 
de développement pour les langages orientés object avec l'utilisation correcte de design pattern
et le respect des règles de base pour la création de logiciels maintenable, performant et fiables.
\newline

D'un point de vu personnel j'ai appris beaucoup de choses sur la gestion d'agenda et les centres d'accueil téléphonique notamment
en terme de contraintes techniques qui sont imposées par des clients réticent au changement 
ainsi que les enjeux technologiques de ce cœur de métier qui doit se faire peau neuve 
pour rester compétitif.

Le marché de la prise de rendez-vous évolue et se tourne de plus en plus vers le 
web et le mobile, \newline
